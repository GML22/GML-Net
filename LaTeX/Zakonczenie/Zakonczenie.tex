\section{Zakończenie}

Niniejsza praca podejmowała problematykę detekcji budynków na zdjęciach lotniczych przy użyciu głębokich sieci neuronowych. Praca ta w~pierwszych rozdziałach skupiała się na zdefiniowaniu najważniejszych pojęć związanych z~głębokim uczeniem, przedstawieniem jego podstaw teoretycznych, historii oraz najistotniejszych architektur z~nim związanych. Kolejne rozdziały zostały poświęcone przeglądowi literatury badawczej z~zakresu badanej problematyki oraz definicji własnej głębokiej sieci neuronowej o~nazwie \textit{GML-Net}. Całość została zwieńczona rozdziałem opisującym sposób generacji finalnych predykcji oraz porównującym uzyskane wyniki do wyników przedstawionych literaturze badawczej.

Zaprezentowana w~bieżącej pracy sieć \textit{GML-Net} pozwoliła na uzyskanie zadowalającej skuteczności w~detekcji budynków na zdjęciach lotniczych pochodzących ze zbioru \textit{Inria Aerial Image Labeling Dataset}. Nie udało się w~prawdzie osiągnąć lepszych rezultatów niż aktualne wyniki \textit{state of art} zaprezentowane w~pracy \textit{Semantic Segmentation from Remote Sensor Data and the Exploitation of Latent Learning for Classification of Auxiliary Tasks} \cite{chatterjee}, ale mimo to wyniki uzyskane przy pomocy sieci \textit{GML-Net} można uznać za satysfakcjonujące. Za duże pole do dalszego rozwoju uzyskanej sieci można uznać sposób łączenia predykcji masek o~rozdzielczości \emph{256x256} w~jedną łączną maskę o~rozdzielczości \emph{5000x5000}, gdyż w~tym procesie skuteczność predykcji istotnie się pogarszała, szczególnie patrząc przez pryzmat metryki \textit{Intersection over Union}. 

Za duże zalety sieci \textit{GML-Net} można natomiast uznać jej ciekawą architekturę wykorzystująca elementy sieci \textit{ResNet}, \textit{U-Net} oraz \textit{ICT-Net}, a także nowatorską funkcję straty stanowiąca ważoną sumę \textit{Binary Cross-Entropy Loss}, \textit{Dice Loss} oraz \textit{Lovász hinge loss}. Stąd też, za uprawnione wydaje się stwierdzenie, iż zaprezentowana w~niniejszej pracy głęboka sieć neuronowa \textit{GML-Net} wnosi nowe, ciekawe spojrzenie do literatury podejmującej problematykę detekcji budynków na zdjęciach lotniczych.