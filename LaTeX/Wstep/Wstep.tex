\section{Wstęp}

Głębokie uczenie zrewolucjonizowało dziedzinę przetwarzania sygnałów cyfrowych w~niespotykanej dotąd skali. Przed erą głębokich sieci neuronowych komputery nie były w~stanie zbliżyć się do skuteczności z~jaką przeciętny człowiek radzi sobie z~takimi zagadnieniami z~dziedziny cyfrowego przetwarzania sygnałów jak na przykład: rozpoznawanie / segmentacja obiektów na zdjęciach cyfrowych, rozpoznawanie pozy / emocji na podstawie analizy filmu wideo czy detekcja poleceń / słów kluczowych na podstawie nagrania audio. Nadejście ery głębokich sieci neuronowych, której początek można datować na 2006 rok (a intensywny rozwój na czas po 2012 roku), spowodowało nie tylko dorównanie przez komputery, w~wielu zadaniach, średniej ludzkiej skuteczności, ale często istotne przewyższenie jej (ang. \emph{superhuman performance}). Dobrym przykładem ponadludzkiej skuteczności głębokich sieci neuronowych jest konkurs \emph{ImageNet Large Scale Visual Recognition Challenge (ILSVRC)}, który polegał na automatycznym generowaniu podpisów do obrazów znajdujących się w zbiorze \emph{ImageNet}. Jeszcze w~2011~roku zwycięskie rozwiązanie w~tym konkursie charakteryzowało się stopą błędów top-5\footnote{Stopa błędów top-5 to miara stosowana przy ocenie modeli klasyfikacji w~ramach uczenia maszynowego, zakłada ona, że model zakwalifikował dany obraz poprawnie gdy wśród pięciu najbardziej prawdopodobnych kategorii wskazanych przez model znajduje się poprawna kategoria} na poziomie 26\%. Przełomowe rozwiązanie zaproponowane przez Krizhevsky A., Sutskever I. i Hinton G. z~2012~roku \cite{alexnet}, wykorzystujące głębokie uczenie się, wygrało ten konkurs osiągając stopę błędów top-5 na poziomie 15\%. W~2017 roku organizatorzy konkursu \emph{ILSVRC} zapowiedzieli zakończenie jego organizacji (przy najlepszym wyniku top-5 równym 2,251\% uzyskanym przez sieć \emph{SENet}), gdyż rezultaty osiągane na zbiorze testowym przez większość zgłaszanych do konkursu modeli zaczęły istotnie przekraczać wyniki przeciętnego człowieka (koło 5\%) - zapowiedzieli oni jednoczenie organizację dużo bardziej wymagającego konkursu wiążącego się z klasyfikacją trójwymiarowych obiektów.

Generowanie automatycznych podpisów do zdjęć jest jednak zadaniem dużo mniej skomplikowanym niż detekcja obiektów na zdjęciach lotniczych czy też satelitarnych. Jest tak, gdyż zdjęcia lotnicze / satelitarne charakteryzują się o~wiele większą złożonością pod względem skali, jakości, rozdzielczości czy też zaszumienia niż zdjęcia pochodzące z~takich zbiorów jak \emph{ImageNet} a~głównym zadaniem sieci nie jest wybranie jednego z~tysiąca podpisów, ale wskazanie piksel po pikselu, do jakiego obiektu dany piksel z~konkretnego zdjęcia należy. Stąd też rezultaty osiągane przez głębokie sieci neuronowe przy tego typu zadaniach są w~dalszym ciągu dużo gorsze niż rezultaty osiągane przez człowieka. Biorąc pod uwagę stale wzrastającą liczbę zdjęć oraz nagrań wideo rejestrowanych codziennie przez satelity / samoloty / drony, śmiało można stwierdzić, iż istnieje duże zapotrzebowanie / pole do rozwoju i~ulepszeń głębokich metod stosowanych do analizy tego typu danych, tak żeby w przyszłości głębokie sieci neuronowe mogły z~ powodzeniem zastąpić w~tym obszarze człowieka. W~szczególności, istotną kwestią przy analizie zdjęć lotniczych i satelitarnych, jest detekcja budynków, gdyż świadomość dokładnego położenia zabudowań na mapach miast jest niezbędna w~takich dziedzinach życia społeczno-ekonomicznego jak: urbanistyka, socjologia, bezpieczeństwo publiczne, ubezpieczenia majątkowe czy też reagowanie kryzysowe w obliczu klęsk żywiołowych.

Niniejsza praca podejmuje problematykę rozpoznawania budynków na zdjęciach lotniczych przy pomocy głębokich sieci neuronowych, gdzie jako zbiór testowy wykorzystywany jest zbiór \emph{Inria Aerial Image Labeling Dataset}. Sieć neuronowa przedstawiona w~niniejszej pracy korzysta z~architektury typu \emph{U-Net}, jej enkoder jest zasilany wagami modelu \emph{Wide ResNet-50-2} \cite{zagoruyko}, całość łączą bloki typu \emph{Bottleneck} (w~wersji OSNet \cite{zhou}) gwarantujące agregację map cech z~przekroju różnych skal. Aby efektywnie nauczyć się korelacji kanałów przestrzennych, uniknąć nadmiernego dopasowania oraz uzyskać większą efektywność obliczeniową, zaimplementowana sieć neuronowa wykorzystuje punktowo i~wgłębnie separowalne konwolucje. Efektywność głębokiego uczenia się jest gwarantowana przez funkcję straty złożoną z~ważonej sumy: Binary Cross-Entropy Loss, Dice Loss oraz Lovász hinge loss. W~kolejnych rozdziałach niniejszej pracy opisywane są podstawy teoretyczne głębokich sieci neuronowych, analizowana jest literatura dotycząca detekcji budynków na zdjęciach lotniczych i~satelitarnych oraz szczegółowo definiowane są: architektura użytej sieci neuronowej, zbiór danych i~jego transformacje, funkcje strat oraz metody oceny skuteczności działania zaimplementowanej sieci. Całość zakończona jest porównaniem uzyskanych wyników z wynikami przedstawionymi w~literaturze.